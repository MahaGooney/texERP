\title{ERP-gestütztes Supply Chain Management in der Pharmabranche}
\author{Nico Schneider und Janik Derenbach}
\documentclass{article}

\usepackage[ngerman]{babel}
\usepackage[utf8]{inputenc}
\usepackage{color}
\usepackage[a4paper,lmargin={4cm},rmargin={2cm},
tmargin={2.5cm},bmargin = {2.5cm}]{geometry}
\usepackage{amssymb}
\usepackage{amsthm}
\usepackage{graphicx}

\begin{document}
\maketitle
\newpage
\tableofcontents
\newpage
\section{Einleitung}
\subsection{Problemstellung und Relevanz}
Die Pharmabranche hat besondere Anforderungen und regulatorische Herausforderungen beim implementieren von ERP-Systemen zu beachten.
Allerdings können ERP-Implementierungen den pharmazeutischen Unternehmen auch dabei helfen, ihre Supply-Chain transparenter und effizienter zu gestalten.
Aus diesem Grund ist es für die Branche relevant die Möglichkeiten und Risiken zu untersuchen die das Einführen und die Weiterentwicklung von ERP-Systemen ermöglicht. \cite{Gronau}
\subsection{Zielsetzung der Arbeit}
\section{Grundlagen}
\subsection{ERP-Systeme}
\subsection{SCM}
\subsection{Pharmazeutische Industrie}

\section{Herausforderungen bei der Abbildung vom SCM im ERP}
\subsection{Regulatorik und Compliance}
\subsection{Digitale Transformation in der Pharmabranche}
\subsection{Lieferkettentransparenz und Rückverfolgbarkeit der Chargen}
\subsection{Risikomanagement}

\section{Lösungsansätze zur Integration des SCM in ERP für Pharmaunternehmen}
\section{Fazit}
\subsection{}

\bibliographystyle{ieeetr} % Wähle den gewünschten Zitationsstil
\bibliography{Literatur} % Name deiner .bib-Datei (ohne Dateiendung)

\end{document}
